\documentclass[12pt,a4paper]{article}%
% \usepackage{a4wide}
\usepackage{amsmath}
\usepackage{graphicx}
\usepackage{alltt}
\usepackage{verbatim}

% Useful definitions.

\newcommand{\finallongpage}{\enlargethispage{\baselineskip}}
\newcommand{\finalshortpage}{\enlargethispage{-\baselineskip}}
\newcommand{\finalpagebreak}{\pagebreak}
\newcommand{\finalnewpage}{\newpage}

% Some characters.
\newcommand{\BS}{\char'134}
\newcommand{\HAT}{\char'136}
\newcommand{\TILDE}{\char'176}

% Some other useful symbols
\newcommand{\bottom}{\ensuremath{\perp}\xspace}
\newcommand{\WRa}{{$\;\;\Rightarrow\;\;$}}

% Visible comments.
\newcommand{\reminder}[1]{%
  \begin{description}%
  \item{\textbf{!!!}} #1%
  \end{description}%
}

\newcommand{\topic}[1]{%
  \begin{description}%
  \item{\textbf{TOPIC}} #1%
  \end{description}%
}

% New environments.

% Old verbatim code: why the extra vspace???
%\newenvironment{code}%
%  {\quote\verbatim}%
%  {\endverbatim\endquote\vspace{\topsep}}

% Verbatim code
\newenvironment{code}%
  {\begin{quote}\begin{alltt}}%
  {\end{alltt}\end{quote}}

% Centered figure
\newenvironment{cfig}%
  {\begin{figure}\centering\setlength{\fboxsep}{0.3cm}}%
  {\end{figure}}

% Theorem-like environments.
% \theoremstyle{plain}

% Theorems
% \newtheorem{theorem}{Theorem}[chapter]

% Definitions
% {
% \theorembodyfont{\rmfamily}
% \newtheorem{definition}{Definition}[section]
% }

% Commands for mathematical typesetting.
\newcommand{\vect}[1]{\mathbf{#1}}			% vector
\newcommand{\ddt}[1]{\frac{\mathrm{d}#1}{\mathrm{d}t}}	% d/dt

% Commands for typesetting lambda expressions.
\newcommand{\abstr}[2]{\texttt{($\lambda$#1.#2)}}
\newcommand{\app}[2]{\texttt{#1~#2}}
\newcommand{\appii}[3]{\texttt{#1~#2~#3}}
\newcommand{\appiii}[4]{\texttt{#1~#2~#3~#4}}
\newcommand{\appiv}[5]{\texttt{#1~#2~#3~#4~#5}}
\newcommand{\paren}[1]{\texttt{(#1)}}
\newcommand{\fundef}[3]{\texttt{#1~#2~=~#3}}
\newcommand{\fundefii}[4]{\texttt{#1~#2~#3~=~#4}}
\newcommand{\fundefiii}[5]{\texttt{#1~#2~#3~#4~=~#5}}

% Typesetting CFGs and derivations:
\newcommand{\nterm}[1]{$\mathit{#1}$}
\newcommand{\term}[1]{\texttt{\textbf{#1}}}
\newcommand{\metaterm}[1]{$\underline{\mathit{#1}}$}
\newcommand{\yields}{$\to$}
\newcommand{\alt}{\; $|$ \;}

\newcommand{\dderives}{\Rightarrow}
\newcommand{\dderivesIn}[1]{\underset{#1}{\Rightarrow}}
\newcommand{\dderiveslm}{\underset{\mathit{lm}}{\Rightarrow}}
\newcommand{\dderivesrm}{\underset{\mathit{rm}}{\Rightarrow}}
\newcommand{\dderivesrmr}{\underset{\mathit{rm}}{\Leftarrow}}
\newcommand{\derives}{\stackrel{*}{\Rightarrow}}
\newcommand{\derivesIn}[1]{\stackrel{*}{\underset{#1}{\Rightarrow}}}
\newcommand{\deriveslm}{\stackrel{*}{\underset{\mathit{lm}}{\Rightarrow}}}
\newcommand{\derivesrm}{\stackrel{*}{\underset{\mathit{rm}}{\Rightarrow}}}


% % Commands for typesetting EBNF and attribute grammars.
% \newcommand{\ra}{$\rightarrow$}
% \newcommand{\alt}{$\mid$ }
% \newcommand{\opt}[1]{[ #1 ]}
% \newcommand{\many}[1]{\{ #1 \}}
% \newcommand{\manyp}[1]{\{ #1 \}$^{+}$}
% \newcommand{\diff}[2]{#1\raisebox{-.6ex}{\small $\langle$#2$\rangle$}}
% \newcommand{\nt}[1]{\textit{#1}}
% \newcommand{\nti}[2]{\textit{#1}$^{#2}$}
% \newcommand{\ntii}[3]{\textit{#1}$^{(#2,#3)}$}
% \newcommand{\trm}[1]{\texttt{#1}}
% \newcommand{\eq}{$=$}
% \newcommand{\iattr}[2]{\nt{#1}$\downarrow$\textit{#2}}
% \newcommand{\sattr}[2]{\nt{#1}$\uparrow$\textit{#2}}

% % Including pictures.
% \newcommand{\gpic}[1]{\mbox{\input{#1}\raise 1em\box\graph}}
% \newcommand{\gpicbox}[1]{\fbox{\input{#1}\raise 1em\box\graph}}

% % Figure X.Y (and table) in bold.
% \makeatletter
% \long\def\@makecaption#1#2{%
%   \vskip\abovecaptionskip
%   \sbox\@tempboxa{\textbf{#1}: #2}%
%   \ifdim \wd\@tempboxa >\hsize
%     \textbf{#1}: #2\par
%   \else
%     \global \@minipagefalse
%     \hb@xt@\hsize{\hfil\box\@tempboxa\hfil}%
%   \fi
%   \vskip\belowcaptionskip}
% \makeatother


\begin{document}

\frenchspacing

\title{G53CMP Compilers: Coursework 1 \\
{\Large 20/10/2014}}
\author{Ryan Shaw\\rxs62u}
\date{\today}
\maketitle

\newpage

\appendix

\section{Task I.1}
\label{app:mg}

\subsection{Grammar extension}

First all the grammars are extended to include the new loop construct
\subsubsection{Lexical Syntax}
\begin{center}
\begin{tabular}{lcl}

\nterm{Keyword} & \yields & \term{begin} $|$ \term{const} $|$ \term{do}
                            $|$ \term{else} $|$ \term{end} $|$ \term{if}
                            $|$ \term{in} \\
                        & & $|$ \term{let} $|$ \term{then} $|$ \term{var} 
                            $|$ \term{while} $|$ \term{repeat} $|$ \term{until}\\
&& \\
\end{tabular}
\end{center}
\subsubsection{Context-Free Syntax}

\begin{center}
\begin{tabular}{lcl}
\nterm{Command}           & \yields & \nterm{VarExpression} \term{:=}
                                      \nterm{Expression}                    \\
                          & \alt    & \nterm{VarExpression} \term{(}
                                      \nterm{Expressions} \term{)}          \\
                          & \alt    & \term{if} \nterm{Expression}
                                      \term{then} \nterm{Command}	    \\
                                   && \term{else} \nterm{Command}           \\
                          & \alt    & \term{while} \nterm{Expression}
                                      \term{do} \nterm{Command}             \\
                          & \alt    & \term{let} \nterm{Declarations}
                                      \term{in} \nterm{Command}             \\
                          & \alt    & \term{begin} \nterm{Commands}
                                      \term{end}                            \\
                          & \alt    & \term{repeat} \nterm{Command} \term{until}
                          			  \nterm{Expression}                    \\
&& \\
\end{tabular}
\end{center}

\subsubsection{Abstract Syntax}
\begin{center}
\begin{tabular}{lcll}
\nterm{Command} & \yields & \nterm{Expression} \term{:=}
			    \nterm{Expression}			& CmdAssign \\
		& $|$	  & \nterm{Expression}
			    \term{(}
			    \nterm{Expression}$^*$
			    \term{)}				& CmdCall   \\
		& $|$     & \term{begin}
			    \nterm{Command}$^*$
			    \term{end}				& CmdSeq    \\
		& $|$     & \term{if} \nterm{Expression}
			    \term{then} \nterm{Command}                     \\
			 && \term{else} \nterm{Command}		& CmdIf	    \\
		& $|$     & \term{while} \nterm{Expression}
			    \term{do} \nterm{Command}		& CmdWhile  \\
		& $|$     & \term{let} \nterm{Declaration}$^*$
			    \term{in} \nterm{Command}		& CmdLet   \\
		& $|$   & \term{repeat} \nterm{Command} 
				\term{until} \nterm{Expression} & CmdRepeat
	 \\
\end{tabular}
\end{center}


% BNF grammar similar to the one given in pages.\ 8--9 of the book

Non-terminals are typeset in italics, like \nterm{this}. Terminals
are typeset in typewriter font, like \term{this}. Terminals whose spelling
(the concrete character sequence) is different from what is shown in the
grammar are typeset in italics and underlined, like \metaterm{Identifier} and
\metaterm{IntegerLiteral}. Their spelling is defined by the lexical
grammar (where they are non-terminals!).

\begin{center}
\begin{tabular}{lcl}
\nterm{Program}           & \yields & \nterm{Command}                       \\
&& \\		          
		          
\nterm{Commands}          & \yields & \nterm{Command}                       \\
                          & \alt    & \nterm{Command} \term{;}
                                      \nterm{Commands}                      \\
&& \\		          
		          
\nterm{Command}           & \yields & \nterm{VarExpression} \term{:=}
                                      \nterm{Expression}                    \\
                          & \alt    & \nterm{VarExpression} \term{(}
                                      \nterm{Expressions} \term{)}          \\
                          & \alt    & \term{if} \nterm{Expression}
                                      \term{then} \nterm{Command}	    \\
                                   && \term{else} \nterm{Command}           \\
                          & \alt    & \term{while} \nterm{Expression}
                                      \term{do} \nterm{Command}             \\
                          & \alt    & \term{let} \nterm{Declarations}
                                      \term{in} \nterm{Command}             \\
                          & \alt    & \term{begin} \nterm{Commands}
                                      \term{end}                            \\
&& \\

\nterm{Expressions}       & \yields & \nterm{Expression}                    \\
                          & \alt    & \nterm{Expression} \term{,}
                                      \nterm{Expressions}                   \\
&& \\

\nterm{Expression}        & \yields & \nterm{PrimaryExpression}             \\
                          & \alt    & \nterm{Expression}
				      \nterm{BinaryOperator}
                                      \nterm{Expression}                    \\
&& \\

\nterm{PrimaryExpression} & \yields & \metaterm{IntegerLiteral}             \\
                          & \alt    & \nterm{VarExpression}                 \\
                          & \alt    & \nterm{UnaryOperator}
                                      \nterm{PrimaryExpression}             \\
                          & \alt    & \term{(} \nterm{Expression} \term{)}  \\
&& \\

\nterm{VarExpression}     & \yields & \metaterm{Identifier}                 \\
&& \\

\nterm{BinaryOperator}    & \yields & \term{\HAT}
			              $|$ \term{*} $|$ \term{/}
                                      $|$ \term{+} $|$ \term{-}
                                      $|$ \term{<} $|$ \term{<=}
                                      $|$ \term{==} $|$ \term{!=}
                                      $|$ \term{>=} $|$ \term{>}
                                      $|$ \term{\&\&} $|$ \term{||}         \\
&& \\

\nterm{UnaryOperator}	  & \yields & \term{-} $|$ \term{!}
\end{tabular}
\end{center}

\begin{center}
\begin{tabular}{lcl}
\nterm{Declarations}      & \yields & \nterm{Declaration}                   \\
                          & \alt    & \nterm{Declaration} \term{;}
                                      \nterm{Declarations}                  \\
&& \\

\nterm{Declaration}       & \yields & \term{const} \metaterm{Identifier}
                                      \term{:} \nterm{TypeDenoter}   
                                      \term{=} \nterm{Expression}          \\
                          & \alt    & \term{var} \metaterm{Identifier}
                                      \term{:} \nterm{TypeDenoter}         \\
                          & \alt    & \term{var} \metaterm{Identifier}
                                      \term{:} \nterm{TypeDenoter}
                                      \term{:=} \nterm{Expression}         \\
&& \\

\nterm{TypeDenoter}  & \yields & \metaterm{Identifier}
\end{tabular}
\end{center}

Note that the productions for \nterm{Expression} makes the grammar as stated
above ambiguous. Operator precedence and associativity for the \emph{binary}
operators as defined in the following table is used to disambiguate:
\begin{center}
    \begin{tabular}{|@{\ttfamily}c|c|c|}
    \hline
    Operator & Precedence & Associativity \\
    \hline
    \HAT            & 1 & right \\
    * /             & 2 & left  \\
    + -             & 3 & left  \\
    < <= == != >= > & 4 & non   \\
    \&\&            & 5 & left  \\
    ||              & 6 & left  \\
    \hline
    \end{tabular}
\end{center}
A precedence level of 1 means the highest precedence, 2 means second
highest, and so on.

Also note that the syntactic category \nterm{Declaration} includes
\emph{definition} of constants, as not only the type is given, but also the
value of the constant (through an expression), and that a variable
declaration includes an optional initialization.

% \newpage

\subsection{MiniTriangle Abstract Syntax}
\label{app:mt-as}

% MiniTriangle Abstract syntax grammar, freely after page 13 of the book

This is the MiniTriangle abstract syntax. It captures the tree structure of
MiniTriangle programs as concisely as possible. It is used as the basis for
designing the datatypes for representing MiniTriangle programs. For example,
note that there is only one non-terminal for expressions as opposed to three
in the grammar for the concrete syntax. The extra non-terminals (along with
a specification of binary operator associativity and precedence) are needed to
make the concrete syntax \emph{unambiguous}, which is necessary for parsing. 
But once a program has been successfully parsed, its structure has been
determined, and ambiguity is no longer an issue. Another difference is that
general function application has been introduced as one expression form. This
replaces both concrete unary operator application and concrete (infix) binary
operator application, as such operators \emph{are} functions of one and two
arguments, respectively. (The concrete syntax of MiniTriangle currently does
not include general function application, but that could easily be added.) As
a consequence, a single ``variable'' terminal \metaterm{Name} also replaces
both \metaterm{Identifier} and \metaterm{Operator}; i.e.,
\metaterm{Identifier} $\subseteq$ \metaterm{Name} and \metaterm{Operator}
$\subseteq$ \metaterm{Name}.

The rightmost column gives the node labels for drawing abstract syntax trees. 
(They correspond to the names of the data constructors of the of the abstract
syntax datatypes in the compiler.) Note that some elements of concrete syntax,
such as keywords, do occur in the productions. They are there to make the
connection between the concrete and abstract syntax clear, and to provide an
\emph{alternative} textual representation for the abstract syntax (e.g. for
use in typing rules). However, these fragments of concrete syntax are
\emph{omitted} when drawing abstract syntax trees, as they are implied by the
node labels and thus superfluous. Also note that some of the productions make
use of the EBNF *-notation for sequences. When drawing an abstract syntax
tree, that means that the corresponding nodes will have a varying number of
children.

% The same typesetting conventions are used as for the grammar for the
% concrete syntax.

% Non-terminals are typeset in normal roman font, like this. Terminals
% are typeset in typewriter font, like \term{this}. Terminals whose spelling
% (the concrete character sequence) is different from what is shown in the
% grammar are typeset in italics, like \metaterm{Identifier} and
% \metaterm{Integer-Literal}. Their spelling is defined by the lexical
% grammar (where they are non-terminals!).


\begin{center}
\begin{tabular}{lcll}
\nterm{Program} & \yields & \nterm{Command} & Program \\
&&& \\

\nterm{Command} & \yields & \nterm{Expression} \term{:=}
			    \nterm{Expression}			& CmdAssign \\
		& $|$	  & \nterm{Expression}
			    \term{(}
			    \nterm{Expression}$^*$
			    \term{)}				& CmdCall   \\
		& $|$     & \term{begin}
			    \nterm{Command}$^*$
			    \term{end}				& CmdSeq    \\
		& $|$     & \term{if} \nterm{Expression}
			    \term{then} \nterm{Command}                     \\
			 && \term{else} \nterm{Command}		& CmdIf	    \\
		& $|$     & \term{while} \nterm{Expression}
			    \term{do} \nterm{Command}		& CmdWhile  \\
		& $|$     & \term{let} \nterm{Declaration}$^*$
			    \term{in} \nterm{Command}		& CmdLet    \\
	&&& \\

\nterm{Expression} & \yields & \metaterm{IntegerLiteral} 	& ExpLitInt \\
		   & $|$     & \metaterm{Name}		 	& ExpVar    \\
		   & $|$     & \nterm{Expression}
			       \term{(}
			       \nterm{Expression}$^*$
			       \term{)}			 	& ExpApp    \\
&&& \\

\nterm{Declaration} & \yields & \term{const}
				\metaterm{Name}
				\term{:}
				\nterm{TypeDenoter}			    \\
			     && \term{=}
				\nterm{Expression}		& DeclConst \\
		    & $|$     & \term{var}
				\metaterm{Name}
				\term{:}
				\nterm{TypeDenoter}			    \\
			     && ( \term{:=} \nterm{Expression}
				$|$ $\epsilon$ )		& DeclVar   \\
&&& \\

\nterm{TypeDenoter} & \yields & \metaterm{Name}			& TDBaseType
\end{tabular}
\end{center}

% \newpage
% 
% \section{MiniTriangle Type System}
% \label{app:ts}
% 
% \newpage
% 
% \section{Notes on the Haskell MiniTriangle Compiler (HMTC)}
% \label{app:hmtc}
% 
% \subsection{Overview of HMTC}
% 
% \subsection{Naming Conventions}



\end{document}

